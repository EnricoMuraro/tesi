%!TEX root = ../dissertation.tex
\chapter{Conclusioni}
\label{conclusion}

Lo stage si è svolto seguendo ogni settimana il piano di lavoro descritto nella sezione \ref{PianoDiLavoro}. Il lavoro è stato ben pianificato dal tutor aziendale e in ogni parte del progetto non ci sono stati ritardi nelle consegne del software o dei documenti. Il tempo, infatti, è stato sufficiente anche per studiare le tecnologie utilizzate e le possibili alternative, per poterle confrontare ed effettuare la scelta migliore. Il codice segue i vincoli di stile scelti ed è stato progettato per essere il più possibile scalabile ed estendibile. Ogni classe e metodo è stato commentato, dai commenti è stata generata automaticamente la documentazione attraverso Sphinx. Grazie ai documenti prodotti e alla documentazione del codice il progetto è facilmente utilizzabile ed estendibile dai ricercatori dell'azienda. Tutti i requisiti individuati sono stati soddisfatti.

Le ultime due settimane, come da piano di lavoro, sono servite come introduzione all'intelligenza artificiale ed è stato possibile studiare vari algoritmi di feature extraction e di anomaly detection. Per addestrare un algoritmo di machine learning in grado di effettuare anomaly detection efficacemente su un problema ampio e complesso come la misurazione dei flussi multifase\cite{multiphaseIntroduction}, sarebbero stati necessari mesi di studio e di lavoro aggiuntivo, chiaramente fuori dai tempi dello stage. I risultati ottenuti sono comunque un ottimo studio di fattibilità sul problema, anche se sono state utilizzate solo alcune delle variabili lette dai sensori e non sono stati ancora tenuti in considerazione i possibili regimi di scorrimento del flusso descritti in sezione \ref{multiphaseflowmeter}. 

Un approccio diverso al problema potrebbe essere effettuare dei test in cui viene manomesso o modificato intenzionalmente il Multiphase Flow Meter per simulare un'anomalia. In questo modo i dati possono essere classificati come "normali" quando sono letti con il MPFM funzionante e come "anormali" quando sono letti con il MPFM modificato. Avere i dati già classificati permette di utilizzare algoritmi di machine learning basati sull'apprendimento supervisionato e ottenere un risultato probabilmente più preciso.
