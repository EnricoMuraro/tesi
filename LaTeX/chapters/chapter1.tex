%!TEX root = ../dissertation.tex
% For an example of a full page figure, see Fig.~\ref{fig:myFullPageFigure}.

\chapter{Contesto aziendale}

\section{L'azienda}
Lo stage si è svolto presso l'azienda Pietro Fiorentini S.p.A nella sede principale di Arcugnano. Pietro Fiorentini realizza prodotti e servizi tecnologicamente avanzati per la distribuzione e l'utilizzo del petrolio e del gas naturale a livello globale, con undici stabilimenti nel mondo.
L'azienda realizza prodotti come valvole, filtri, regolatori di pressione, ma lo stage si è interessato in particolare sui Multiphase Flow Meter.

\section{Multiphase Flow Meter}
Un Multiphase Flow Meter, o MPFM in breve, è un dispositivo utilizzato per misurare le singole portate delle fasi costitutive di un determitato flusso. In questo caso il flusso è una miscela di petrolio, acqua e gas creata durante il processo di estrazione dal pozzo petrolifero.
I MPFM prodotti dalla Pietro Fiorentini sono strumenti non intrusivi per effettuare misurazioni in tempo reale del flusso  evitando l'uso di sistemi basati sulla separazione delle diverse fasi.

Gli MPFM sono modulari e possono essere installati in diverse configurazioni, ogni modulo si occupa di effettuare una o più misurazioni sul flusso.

\section{Struttura dei dati} \label{"StrutturaDeiDati"}
I dati rilevanti al progetto riguardano una serie di test effettuati in vari laboratori negli ultimi dieci anni. I test consistono nell'uso di un Multiphase Flow Meter in un circuito chiuso, in modo da poter controllare accuratamente la composizione e le condizioni del flusso che viene fatto circolare. 

Durante il test si registrano le letture di tutti i sensori presenti nel MPFM, questi dati vengono quindi salvati in file singoli, ognuno dei quali contiene in genere un minuto di lettura. I file sono in formato proprietario BIN o BIX e rappresentano i dati nel loro stato "raw", ad esempio un sensore che misura la pressione non salverà direttamente dei valori in atmosfere ma potrebbe salvare dei voltaggi che dovranno essere a loro volta interpretati.

Le informazioni sulla configurazione del MPFM e sulle condizioni del flusso non sono salvate nei file raw, sono salvate invece in file separati, rispettivamente un file di testo per la configurazione del MPFM e un foglio di calcolo per le condizioni del flusso. Questo foglio di calcolo, chiamato anche foglio di riferimento, rappresenta quindi le condizioni in cui sono state effettuate le letture dei sensori, in particolare descrive le seguenti proprietà del flusso per ogni file raw:

\begin{itemize}
	\item \bfseries{Qgas}: Il volume della componente di gas nel flusso
	\item \bfseries{Qwater}: Il volume della componente d'acqua nel flusso
	\item \bfseries{Qwater}: Il volume della componente di petrolio nel flusso
	\item \bfseries{WLR}: Water Liquid Ratio, la percentuale di acqua sul totale del liquido. Calcolato con la formula: \[\frac{Qwater}{Qwater+Qoil}\]
	\item \bfseries{GVF}: Gas Volume Fraction, la frazione di gas sul volume totale. Calcolato con la formula: \[\frac{Qgas}{Qwater+Qoil+Qgas}\]
	\item \bfseries{Pressure}: La pressione del flusso
	\item \bfseries{Temperature}: La temperatura del flusso
\end{itemize}




%% Requires fltpage2 package
%%
% \begin{FPfigure}
% \includegraphics[width=\textwidth]{figures/fullpage}
% \caption[Short figure name.]{This is a full page figure using the FPfigure command. It takes up the whole page and the caption appears on the preceding page. Its useful for large figures. Harvard's rules about full page figures are tricky, but you don't have to worry about it because we took care of it for you. For example, the full figure is supposed to have a title in the same style as the caption but without the actual caption. The caption is supposed to appear alone on the preceding page with no other text. You do't have to worry about any of that. We have modified the fltpage package to make it work. This is a lengthy caption and it clearly would not fit on the same page as the figure. Note that you should only use the FPfigure command in instances where the figure really is too large. If the figure is small enough to fit by the caption than it does not produce the desired effect. Good luck with your thesis. I have to keep writing this to make the caption really long. LaTex is a lot of fun. You will enjoy working with it. Good luck on your post doctoral life! I am looking forward to mine. \label{fig:myFullPageFigure}}
% \end{FPfigure}
% \afterpage{\clearpage}
