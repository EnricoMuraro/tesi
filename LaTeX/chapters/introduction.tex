%!TEX root = ../dissertation.tex
\chapter{Introduzione}
\label{introduction}

Il machine learning è una branca dell'intelligenza artificiale sempre più di interesse per le aziende, anche per quelle che non riguardano direttamente il campo dell'informatica.

Le applicazioni di machine learning sono già molto numerose, e spesso vengono utilizzate senza nemmeno rendersene conto.
Ad esempio viene utilizzato per il riconoscimento vocale o identificazione della scrittura manuale, dai motori di ricerca per mostrare i risultati più rilevanti, oppure nel settore della ricerca scientifica in campo medico dove gli algoritmi imparano a effettuare diagnosi di tumori o altre malattie prima che diventino un problema. Un'altra delle possibili applicazioni per il machine learning è il riconoscimento delle anomalie nel funzionamento di uno strumento o macchinario.

Avere sotto controllo lo stato di salute dei propri strumenti è di grande interesse per qualunque azienda, e con il machine learning è possibile non solo ottenere una diagnostica automatica delle anomalie, ma anche prevedere la rottura dello strumento prima che possa causare ulteriori danni.

Tutto questo è possibile solo con una grande quantità di dati sul problema che si vuole affrontare, sia per il riconoscimento vocale, sia per l'autodiagnostica di un macchinario. Meno sono i dati disponibili meno risulterà efficace ed utile l'applicazione di algoritmi di machine learning