%!TEX root = ../dissertation.tex

\chapter{Il progetto di stage}
Pietro Fiorentini sta sviluppando un sistema di auto-diagnostica per i Multiphase Flow Meter, con l'obiettivo di avere sistemi sempre più intelligenti che riescano a dare informazioni relativamente al loro stato di funzionamento e salute. 
I dati raccolti da vari sensori presenti nei MPFM sono analizzati attraverso algoritmi di machine learning per identificare automaticamente eventuali anomalie o malfunzionamenti.
Il progetto include la creazione di adeguate infrastrutture software per la gestione dei dati e la loro analisi, per poi poter applicare e valutare gli algoritmi di machine learning.

\section{Piano di lavoro}
Lo stage segue un piano di lavoro redatto prima dell'inizio dello stage dal tutor aziendale in collaborazione con lo stagista e approvato dal tutor universitario.
Il piano di lavoro definisce gli obiettivi di ogni settimana di stage.

\subsection{Settimana 1 - Analisi dei requisiti}

\begin{itemize}
	\item Incontro con persone coinvolte nel progetto Machine Learning per discutere i requisiti e le richieste relativamente al progetto e alla qualità del software
	\item Verifica credenziali e strumenti di lavoro assegnati
	\item Presa visione dell'infrastruttura esistente
	\item Analisi dei requisiti, tenendo in considerazione il preesistente
	\item Discussione e redazione del documento del "Requirement Specification"
\end{itemize}
Risultato: Documento "Requirement Specification" che raccoglie tutti i vincoli e requisiti del progetto.

\subsection{Settimana 2 e 3 - Gestione dei dati}

\begin{itemize}
	\item Studio dei dati e delle informazioni a corredo (quando sono stati raccolti, dove, con quale strumento)
	\item Studio delle infrastrutture per gestire i dati (DB, Data Warehouse, Data Lake, etc)
	\item Creare struttura per gestire e  accedere facilmente ai dati, combinando tutte le informazioni disponibili
\end{itemize}
Risultato: Documento "Data Management Design" che spiega come implementare l'infrastruttura software per la gestione dei dati.

\subsection{Settimana 4 e 5 - Analisi dei dati}

\begin{itemize}
	\item Studio del formato con cui sono salvati i dati (non standard, proprietario)
	\item Librerie in Python per accedere al contenuto dei dati e poterli analizzare
	\item Creare la struttura software per estrarre le informazioni dai dati e per creare grafici e analizzarli
	\item Strumento o interfaccia per visualizzare e analizzare dati ricavati da fonti diverse
\end{itemize}
Risultato: Documento "Data Analytics Design" che descriva come implementare la lettura dei dati grezzi, creare grafici e analizzarli, creazione del software per interfacciarsi con i dati.

\subsection{Settimana 6 - Feature extraction}

\begin{itemize}
	\item Studio di metodi per estrazione di caratteristiche dai dati e metriche per il confronto dei metodi, con lo scopo di rendere più semplice l’individuare anomalie
	\item Implementazione e confronto dei vari metodi per estrarre caratteristiche
\end{itemize}
Risultato: Documento "Feature Extraction Design \& Report" che descriva i metodi di estrazione e il loro confronto e valutazione, creazione del software con i metodi implementati.

\subsection{Settimana 7 - Anomaly detection}

\begin{itemize}
	\item Studio di modelli e algoritmi di machine learning per la rilevazione di anomalie, studio di metriche per il confronto
	\item Valutazione e confronto dei modelli sui dati per identificare anomalie sulla base di metriche prestabilite
\end{itemize}
Risultato: Documento "Anomaly Detection Design \& Report" che descriva gli algoritmi di identificazione anomalie, il loro confronto e valutazione, creazione del software con gli algoritmi implementati

\subsection{Settimana 8 - Conclusione}

\begin{itemize}
	\item Incontro con il team di Ricerca e Sviluppo per presentare e insegnare ad usare quanto prodotto
	\item Live-demo di tutto il lavoro di stage
\end{itemize}
Risultato: Presentazione e live-demo

\section{Obiettivi}

\subsection{Obiettivi minimi}
Documenti:
\begin{itemize}
	\item Requirement Specification: sezione "General Requirement"
	\item Data Management Design: diagramma a blocchi dell'infrastruttura
	\item Data Analytics Design: diagramma a blocchi dell'infrastruttura
	\item Feature Extraction Design \& Report: descrizione di un solo metodo di
	feature extraction
	\item Anomaly Detection Design \& Report: due algoritmi di anomaly detection descritti e valutati
	\item Presentazione finale
\end{itemize}
Software:
\begin{itemize}
	\item Database di gestione dei dati
	\item Software per l'analisi dei dati (script per accedere ai dati)
	\item Software per feature extraction, un solo metodo implementato
	\item Software per anomaly detection, implementazione, confronto e valutazione di due algoritmi
\end{itemize}

\subsection{Obiettivi desiderabili}
Tutti gli obiettivi minimi con le seguenti aggiunte o modifiche.

Documenti:
\begin{itemize}
	\item Requirement Specification completo di informazioni per verificare e testare i
	requisiti
	\item Data Management Design completo
	\item Data Analytics Design completo
	\item Feature Extraction Design \& Report: descrizione di tre metodi di feature extraction, valutazione e confronto
	\item Anomaly Detection Design \& Report: descrizione di cinque algoritmi di anomaly detection, valutazione e confronto
	\item Presentazione finale e dimostrazione live del software sviluppato su un set di dati
\end{itemize}
Software:
\begin{itemize}
	\item Infrastruttura di gestione dati completamente implementata
	\item Software per l'analisi dei dati completamente implementato
	\item Software per feature extraction, confronto e valutazione di tre metodi
	\item Software per anomaly detection, confronto e valutazione di cinque algoritmi
\end{itemize}


\section{Vincoli}
Il progetto deve rispettare dei determinati vincoli imposti dall'azienda o dall'università

\subsection{Vincoli temporali}
L'università impone un vincolo temporale minimo di 300 ore di lavoro durante lo stage. Per questo motivo il progetto è stato diviso in 8 settimane, lavorando 40 ore a settimana per un totale massimo di 320 ore.

\subsection{Vincoli metodologici}
I documenti, il codice e i commenti scritti durante il progetto devono essere completamente in inglese.

Tutto ciò che viene prodotto durante lo stage deve essere versionato nel repository Git fornito dall'azienda.

L'azienda non ha definito altri vincoli metodologici, e ha dato libera scelta per gli strumenti di sviluppo. Come IDE è stato utilizzato PyCharm essendo uno degli IDE più completi per lo sviluppo in Python. GitKraken è stato usato come client di Git per facilitare il lavoro condiviso attraverso il repository fornito. Per la documentazione del codice è stato usato Sphinx, in modo da generare la documentazione automaticamente a partire dai commenti presenti nel codice.

\subsection{Vincoli tecnologici}
L'utilizzo di Python 3 come linguaggio di programmazione è stato fortemente consigliato dall'azienda vista la presenza di molte librerie per la visualizzazione dei dati e l'applicazione di feature extraction e anomaly detection.

La scelta del database è stata libera, ma MySql è stato preferito in quanto già installato e funzionante all'interno dell'azienda.

